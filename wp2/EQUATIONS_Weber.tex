\documentclass[12pt,a4paper]{article}

\usepackage[OT1]{fontenc}
\usepackage[utf8]{inputenc}
\usepackage[russian,english]{babel}
%%%% Additional part of preamble generated by package Sweave in RStudio
%% initial declaration for fonts
\usepackage[labelfont=bf]{caption}
\usepackage[font=small,labelfont=bf]{subcaption}

% for testing
\usepackage{lipsum}

%% For tables
\usepackage{array}
\usepackage{longtable}
\usepackage{fullpage}
\usepackage{dcolumn}
\usepackage[flushleft]{threeparttable}
\usepackage{booktabs}
\usepackage[12hr]{datetime}
\usepackage[DIV=16]{typearea}
\usepackage{scrextend,booktabs}
\usepackage{tabulary}
\usepackage{multirow}
\usepackage[font=footnotesize]{caption}
\usepackage{lscape}

%% Additional elements for figures and numbering added by Sweave
\usepackage{float}
\usepackage[sc]{mathpazo}
\usepackage{geometry}
\geometry{verbose,tmargin=2.5cm,bmargin=2.5cm,lmargin=2.5cm,rmargin=2.5cm}
\setcounter{secnumdepth}{2}
\setcounter{tocdepth}{2}
\usepackage{url}

%% Additional links for hyperref
\usepackage[unicode=true,pdfusetitle,
bookmarks=true,bookmarksnumbered=true,bookmarksopen=true,bookmarksopenlevel=2,
breaklinks=false,pdfborder={0 0 1},backref=false,colorlinks=false]
{hyperref}
\hypersetup{
	pdfstartview={XYZ null null 1}}

\usepackage{breakurl}
\usepackage{Sweave}

%% For math equations
\usepackage{amsmath}
\numberwithin{equation}{section}

%% For bookmarks
\usepackage{bookmark} 

\usepackage[backend=bibtex,
natbib=true, 
style=chicago-authordate]{biblatex}
\addbibresource{Returns.bib}


%% For figures numbered by section
\usepackage{chngcntr}
\counterwithin{figure}{section}
\counterwithin{table}{section}

%% For color in text
\usepackage{xcolor}

%% For sparklines in table
\usepackage{graphicx} 
\usepackage{lmodern} 
\newcommand{\graph}[3]{
	\raisebox{-#1mm}{\includegraphics[height=#2em,width=3cm]{#3}}
}

%% For tables in panel
\usepackage{ltablex,ragged2e}
\renewcommand\tabularxcolumn[1]{>{\RaggedLeft}p{#1}}

\begin{document}

\subsection{Depreciation of Human Capital using Non-Linear Least Squares}

\subsubsection{Description of Model}

\citet{arrazola_132b._2005} and \citep{weber_173._2008, weber_156._2011} implemented a model that includes the human capital depreciation rate as a parameter in the earnings equation.

Weber starts with the definition of $s_{t}$ -- the time fraction invested into the generation of new human capital by a person at age $t$. Relying on a human capital theory implication about the decline of $s_{t}$ over the life cycle, the scholar shows that the complete path of $s_{t}$ is written as follows:

\begin{flalign}\label{eq:1} 
s_{t}=\left\{\begin{array}{ll}
{0} & {\text { if } t<6} \\
{1} & {\text { if } 6 \leq t<S^{\star}} \\
{\alpha-\frac{\alpha}{T-S^{*}} \cdot\left(t-S^{\star}\right)=\alpha \cdot\left(1-\frac{X_{t}}{L}\right)} & {\text { if } S^{\star} \leq t \leq T}
\end{array}\right.
\end{flalign}

\noindent
where $\alpha$ is a parameter, $S^{*}$ is the age when schooling life ends and the working one begins, $T$ is the retirement age, $L = T - S^{*}$ is the total working life length, $X_{t} = t - S^{*}$ is experience. Schooling duration is equal to $S^{*} - 6$.

We then assume that potential earnings $E_{t}$ are exponentially related to the human capital stock:

\begin{flalign}\label{eq:2} 
E_{t}=W \cdot \exp \left(\beta_{K} K_{t}+\beta_{Z} Z_{t}\right)
\end{flalign}

\noindent
where $W$ is a return per period on a unit of earnings capacity, $K_{t}$  is the stock of human capital at time $t$, $Z_{t}$ is a set of observable attributes supposed to influence on earnings, and $K, Z$ are the parameters of interest.

The stock of human capital in period $t$ can be estimated as the sum of the stock from the previous period minus the loss due to depreciation plus the quantity generated during the $t_{th}$ period:

\begin{flalign}\label{eq:3} 
K_{t}=K_{t-1}-\delta \cdot K_{t-1}+\Delta K_{t}=(1-\delta) \cdot K_{t-1}+\Delta K_{t}
\end{flalign}

By recursion, an expression for $K_{t}$ as a function of the human capital stock acquired at the end of formal education $K_{S}$ is given by:

\begin{flalign}\label{eq:4}
K_{t}=(1-\delta)^{t} \cdot K_{S}+\sum_{j=S^{*}}^{t-1}(1-\delta)^{j} \cdot \Delta K_{t-j}
\end{flalign}

Taking the logarithms of the expression \ref{eq:2} and substituting $K_{t}$ by the equation \ref{eq:4} leads to:

\begin{flalign}\label{eq:5} 
\ln E_{t}=\ln W+\beta_{K} \cdot\left\{(1-\delta)^{t} \cdot K_{S}+\sum_{j=S^{*}}^{t-1}(1-\delta)^{j} \cdot \Delta K_{t-j}\right\}+\beta_{Z} Z_{t}
\end{flalign}

Due to the fact that potential earnings are not observable and only a proportion of $s_{t}$  is left to the earnings production, the observed earnings can be expressed by:

\begin{flalign}\label{eq:6} 
\begin{aligned}
&Y_{t}=\left(1-s_{t}\right) \cdot E_{t}\\
&\ln Y_{t}=\ln \left(1-s_{t}\right)+\ln E_{t}
\end{aligned}
\end{flalign}

Combining \ref{eq:5} and \ref{eq:6} results in:

\begin{flalign}\label{eq:7} 
\ln Y_{t}=\ln W+\beta_{K} \cdot\left\{(1-\delta)^{t} \cdot K_{S}+\sum_{j=S^{*}}^{t-1}(1-\delta)^{j} \cdot \Delta K_{t-j}\right\} + \beta_{Z} Z_{t} + \ln \left(1-s_{t}\right)
\end{flalign}

Finally, as the human capital stock at the end of education is related to the human capital received, and we assume a direct association between this stock and the schooling duration:

\begin{flalign}\label{eq:8} 
K_{S}=S
\end{flalign}

The production of new human capital $K_{t}$ depends on the portion of time that is devoted to this activity, therefore we can simplify:

\begin{flalign}\label{eq:9} 
\Delta K_{t}=s_{t}=\left\{\begin{array}{ll}
{0} & {\text { if } t<6} \\
{1} & {\text { if } 6 \leq t<S^{\star}} \\
{\alpha \cdot\left(1-\frac{X_{t}}{L}\right)} & {\text { if } S^{*} \leq t \leq T}
\end{array}\right.
\end{flalign}

Using \ref{eq:3} and \ref{eq:6} to express $K_{S}$ as a sum of the human capital quantities produced during schooling, we get:

\begin{flalign}\label{eq:10} 
K_{S} \stackrel{(3)}{=} \sum_{j=0}^{S^{*}}(1-\delta)^{j} \cdot \Delta K_{S^{*}-j} \stackrel{(9)}{=} \sum_{j=6}^{S^{*}}(1-\delta)^{j}
\end{flalign}

Substituting \ref{eq:8} and \ref{eq:9} into \ref{eq:7}, adding an error term and an individual subscript $i$  culminate in the estimated equation:

\begin{multline}\label{eq:11} 
\ln Y_{i t}= \ln W+\beta_{K} \cdot\left\{(1-\delta)^{X_{i t}} \cdot S_{i}+\alpha \cdot \frac{1-(1-\delta)^{X_{i t}}}{\delta}\right.\cdot\\
\left.\left(1+\frac{1-\delta}{\delta \cdot L_{i}}\right)-\frac{\alpha \cdot X_{i t}}{\delta \cdot L_{i}}\right\}+\ln \left\{1-\left(\alpha-\frac{\alpha}{L_{i}} \cdot X_{i t}\right)\right\}+\beta_{Z} \cdot Z_{i t}+u_{i t}
\end{multline}

\noindent
where $t$ shows a time period, $\ln Y$ is a logarithm of the observed earnings, $\ln W$ is a logarithm of a return per certain period on a unit of earnings capacity, $\beta_{K}$ is the effect of the human capital stock on earnings, $\beta_{Z}$ is the effect of other covariates in the model on earning, $\delta$ is the human capital depreciation rate, $X_{i t}$ is the labor market experience, $L_{i}$ is the total working life length, $\alpha$ is a parameter reflecting the share of time invested in training, $Z_{i t}$ is a set of observable attributes hypothesized to have an impact on earnings, $u_{i t}$ is an error term.


\end{document}