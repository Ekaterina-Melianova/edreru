\documentclass[12pt,a4paper]{article}

%% Maybe WB style just a bit of space between paragraphs
\setlength{\parskip}{2pt}

\usepackage[OT1]{fontenc}
\usepackage[utf8]{inputenc}
\usepackage[russian,english]{babel}

%%%% Additional part of preamble generated by package Sweave in RStudio
%% initial declaration for fonts
\usepackage[labelfont=bf]{caption}
\usepackage[font=small,labelfont=bf]{subcaption}
\captionsetup[sub]{font=small,labelfont={bf,sf}}

%% Additional links for hyperref
\usepackage[unicode=true,pdfusetitle,
 bookmarks=true,bookmarksnumbered=true,bookmarksopen=true,bookmarksopenlevel=2,
 breaklinks=false,pdfborder={0 0 1},backref=false,colorlinks=false]
 {hyperref}
\hypersetup{pdfstartview={XYZ null null 1}}


% for testing
\usepackage{lipsum}

%% For tables
\usepackage{array}
\usepackage{longtable}
\usepackage{fullpage}
\usepackage{dcolumn}
\usepackage[flushleft]{threeparttable}
\usepackage{booktabs}
\usepackage[12hr]{datetime}
\usepackage[DIV=16]{typearea}
\usepackage{scrextend,booktabs}
\usepackage{tabulary}
\usepackage{multirow}
\usepackage[font=footnotesize]{caption}
\usepackage{lscape}

%% Additional elements for figures and numbering added by Sweave
\usepackage{float}
\usepackage[sc]{mathpazo}
\usepackage{geometry}
\geometry{verbose,tmargin=2.5cm,bmargin=2.5cm,lmargin=2.5cm,rmargin=2.5cm}
\setcounter{secnumdepth}{2}
\setcounter{tocdepth}{2}
\usepackage{url}

\usepackage{breakurl}
\usepackage{Sweave}

%% For math equations
\usepackage{amsmath}
\numberwithin{equation}{section}

%% For bookmarks
\usepackage{bookmark} 

\usepackage[backend=bibtex,
			natbib=true, 
			style=chicago-authordate]{biblatex}
\addbibresource{Returns.bib}


%% For figures numbered by section
\usepackage{chngcntr}
\counterwithin{figure}{section}
\counterwithin{table}{section}

%% For color in text
\usepackage{xcolor}

%% For sparklines in table
\usepackage{graphicx} 
\usepackage{lmodern} 
\newcommand{\graph}[3]{
\raisebox{-#1mm}{\includegraphics[height=#2em,width=3cm]{#3}}
}

%% For tables in panel
\usepackage{ltablex,ragged2e}
\renewcommand\tabularxcolumn[1]{>{\RaggedLeft}p{#1}}

%% For own made graphs
\usepackage{tikz}

\usepackage{relsize}




\begin{document}

\title{Trends in the returns to education in the Russian Federation: Can depreciation provide an explanation? \footnote{Arranged in alphabetical order of author last names. The entire code used to generate the graphs and tables presented in this paper is available on bitbucket at \burl{https://bitbucket.org/zagamog/edreru/src/master/} , starting from the raw RLMS data graciously provided in the public domain by the National Research University-Higher School of Economics (HSE). Thanks are also due to Sylvain Weber for generously sharing the code from his University of Geneva Doctoral Thesis, which we adapted for one of the reported sets of estimations.}}
\author{Ekaterina Melianova, Suhas D. Parandekar, Harry A. Patrinos, Art\"{e}m Volgin}
\maketitle

\begin{abstract}
This paper reports the estimation of a Mincerian model and traces the evolution of returns to education in the period 1994-2018. The estimates show an increase in the returns in the first half of this period, followed by a gradual decline. The paper explores the aspect of depreciation of human capital, a topic that has recently become salient because of the threat of automation. Depreciation follows a reverse trajectory, decreasing and them increasing, which may explain part of the observed tendency for the returns to education. 
\end{abstract}

\section{Returns to Education in the Russian Federation: Introduction}

\subsection{Motivation}

The following graphs set out the human capital challenge facing the Russian Federation. Figure \ref{fig:1.1a} shows the extent of the gap in per capita human capital wealth between Russia and the OECD and difference in growth rates of per capita wealth in the period 2000-2014. Growth was ten times higher in Russia, though Figure \ref{fig:1.1a} does not show annual rates. 
\begin{center}
	\begin{figure}[H]
		\begin{minipage}[b]{1\linewidth}
			\centering
			\hspace*{-0.7in}
			\includegraphics[width=400pt]{graph_1a.png}
			% plot 1
			\subcaption{Human Capital Levels and Growth Rates: 2001-2014}\label{fig:1.1a}
		\end{minipage}
		\hfill
		\begin{minipage}[b]{1\linewidth}
			\centering
			\hspace*{-0.7in}
			\includegraphics[width=500pt]{graph_1b.png}
			% plot 1
			\subcaption{Peak in Enrollment in University Education (HSE Yearbook)}\label{fig:1.1b}
		\end{minipage}
		\hfill
		\caption{A set of stylized facts about Human Capital in the Russian Federation}\label{fig:1.1}
	\end{figure}
\end{center}

\vspace{-2em}

Per capita wealth in Russia grew at an annual rate of 4.7\% between 2000-2010 and then at a slower annual rate of 1.8\% between 2010-2017, according to the World Bank report. Figure \ref{fig:1.1a}  shows that the per capita human capital wealth level at average for the OECD in 2014 was about USD 500,000 - five times that of Russia's 95,000 (measured in 2014 dollars). If Russia were to revert to the 2000-2010 level of growth of human capital, it would take about fifty years to catch up with the OECD; at the slower rate of growth of 1.8\%, Russia would not be able to catch up even in one hundred years. 

Figure \ref{fig:1.1b}  shows the percentage of 17-25-year-olds in the Russian federation enrolled in university programs. The figure shows that even as university enrollment had expanded very rapidly since 2000, when 23\% of the age group were enrolled, university enrollment appears to have peaked and then declined. Women reached a peak of 41.1\% enrollment in 2010 and had declined to 34.8\% by 2016. Men reached a peak of 30.4\% enrollment in 2013 and have declined more slowly to 28.9\% in 2016. Every 0.1\% of change is roughly equivalent to 120,000 individuals. 

How can we address the challenge of human capital in the Russian Federation? Is it possible to reverse the declining trend of enrollment of women in Higher Education? What policies would support such an outcome? Together with other research being implemented  by the World Bank and by researchers outside of the World Bank, this paper is motivated by the objective of describing a set of evidence based policy recommendations to enhance the human capital wealth of the Russian Federation. 
 
\subsection{Data}

For country level analysis, we use the Russian Longitudinal Monitoring Survey (RLMS) - the only representative Russian survey with a sizable panel component allowing for a dynamic analysis \parencite{kozyreva_081._2015}. The data are notable for their reliability, diversity, and applicability to a variety of research questions. The RLMS embraces information on people's income and expenditure structure, their material well-being, educational and occupational behavior, health state and nutrition, migration, etc.  RLMS sampling procedures have been thoroughly and extensively described elsewhere \parencite{kozyreva_081._2015}. The present research uses all 23 waves (1994 - 2018) that are available as of \today. The sub-sample selected for empirical investigation in this paper consists of working individuals aged 25-64 who are out of school and have positive labor market experience and income. 
\\

\subsection{Estimation}

Our empirical analysis present results for the general working population of the Russian Federation aged 25-64 and by gender. We use a basic Mincerian specification shown in equation \eqref{eq:1.1}: 


\begin{flalign}\label{eq:1.1} 
Log(Wage) = b_0 + b_1\cdot Educ + b_2\cdot Exp + b_3\cdot Exp^2 + b_4\cdot Gender + \epsilon
\end{flalign}



\noindent
where $Log(Wage)$ is a logarithm of monthly wage, $Educ$ stands for the years of education or highest attained level of education, $Exp$ and $Exp^2$ reflect the years of working experience and its quadratic term respectively, $Gender$ is a dummy variable for gender, $b_0$ is an intercept, $b_1 ... b_n$ are the respective slope estimates, $\epsilon$ refers to a normally distributed error term.

%\subsection{Measures}

\subsubsection{Dependent variable}

For the dependent variable, we used the logarithm of an average monthly wage within the past year on a person's primary job (J13.2 variable in the RLMS dataset). If a person had an additional job, the maximum wage value among the two (J13.2 and J40) was selected for the analysis. In the waves from 1994 to 1996, the question mentioned above was absent; for those waves, we exploited a variable about the average amount of money earned by a respondent within the past 30 days (J10) as a reasonable approximation.

\subsubsection{Independent variables}

The present research uses both metric (measured in years) and categorical education variables. The metric version was created by assigning the average expected number of years corresponding to each attained education level. For the categorical version (EDUC), we distinguished three categories: \textit{(1) secondary, (2) vocational, and (3) higher}. Incomplete levels were incorporated into the respective upper categories (e.g., incomplete higher - into higher). We are interested in exploring returns to education in general, and vocational and higher education. Estimations of premiums to primary and secondary schooling levels are technically unreachable to us since the number of adults without primary education, and the number of adults with only primary level is minuscule in the general population. 

The experience variable was calculated as a difference between current age and years of education minus $6$. Gender was included in the specification in the form of a dichotomous dummy variable with "$1$" standing for females, "$0$" - for males.

% \subsection{Findings from Estimation of Mincerian Equation}

Figure \ref{fig:1.2} shows rates of overall and gender-wise returns to education in Russia in 1994-2018: percentage increment in a person's earnings due to one additional year of schooling (exact estimation results available on github site for this paper). Overall, one can notice a moderate curved growth in returns to education in Russia, achieving its peak in the early 2000s, which is followed by a downward pattern. Education payoffs for women are higher than those of men, but the difference appears to have narrowed in recent years (see tables with 95\% confidence intervals of regression coefficients on gihub site for this paper).

\begin{figure}[H]
 \centering
 \includegraphics[width=\textwidth, height=300pt]{re_edu.png}
 \caption{Rates of Returns to Education in Russia, RLMS 1994-2018}\label{fig:1.2}
\end{figure}

We are particularly interested in the returns to specific levels of education, estimated through a series of dummy variables. Using Secondary Education completed as the base or omitted dummy for purpose of interpretation, we use dummies for Vocational Education and for Higher Education. The specification is presented in equation \eqref{eq:1.2}: 

\begin{flalign}\label{eq:1.2} 
Log(Wage) = a_0 + a_1\cdot D_{Voc} + a_2\cdot D_{Higher} + a_3\cdot Exp + a_4\cdot Exp^2 + a_5\cdot Gender + \epsilon
\end{flalign}




Figure \ref{fig:1.3}, panel (a) displays rates of returns to Higher and Vocational education (as compared to Secondary education) in Russia for the period 1994-2018. The results suggest that on average wage premiums to university education in Russia are roughly 3-5 times greater than to vocational schooling. The observed trend for premiums to both Vocational and Higher education levels is similar to the trend for education in general with the following peaks: 83\% for Higher education and 26\% for Vocational education compared to the average earnings of workers with a Secondary education. The interesting pattern to note from panel \ref{fig:1.3a} is the apparent co-movement of vocational education and higher education - the higher education smoothing curve turns a bit more sharply than the one for vocational education, but their movement is matching, even at second-order levels of smoothness. Further, even though higher education premium remains much above the premium for vocational education, there is a perceptible narrowing of the difference in recent years. Panel \ref{fig:1.3b}, which is drawn from a presentation made by Marina Telezhkina at the WB-HSE Summer School on the Economics of Education in July 2019, shows the interesting pattern of higher education enrollment rates for the population of 17-25 year old. Panel \ref{fig:1.3b} shows the downturn in returns reflected in enrollments, with the peak in enrollments coming about 10 years later. 

\begin{figure}[H]
\caption{\textbf{Rates of Returns to Higher and Vocational Education in Russia, RLMS 1994-2018}}\label{fig:1.3}
         \centering
         \begin{subfigure}[b]{0.5\textwidth}
                 \includegraphics[width=\textwidth]{re_HE_all.png}
                 \caption{Rates of Return}
                 \label{fig:1.3a}
         \end{subfigure}%
         ~ %add desired spacing between images, e. g. ~, \quad, \qquad, \hfill etc.
           %(or a blank line to force the subfigure onto a new line)
         \begin{subfigure}[b]{0.5\textwidth}
                 \includegraphics[width=\textwidth]{telezhkina.png}
                 \caption{Enrollment in Higher Education}
                 \label{fig:1.3b}
         \end{subfigure}
     \end{figure}

When estimated separately by gender, we find trend variation by gender. The results from estimation of earnings functions show that payoffs to Higher education for males varied from 45\% to 76\%, whereas women's returns are described by an inversely U-shaped pattern, reaching their maximum of 104\% in 2001. Within roughly the last 5 years, wage premiums to higher education for women have stabilized around the level of men (~50\%).  Gender wise enrollment rates in higher education (not shown) ten years later appears to match the differences in rates of return, strengthening the hypothesis that market rates of return to education in Russia do indeed influence individual continuing school decisions. 

A similar comparative picture is observed with respect to vocational education, albeit with a different kind of variation by gender (see Figure \ref{fig:1.4}): returns for males are almost flat within the time period while returns for females shows a concave pattern. The overall outcome concerning payoffs to schooling isolated by gender has been confirmed in a similar fashion by past studies \parencite[e.g.,][]{cheidvasser_006._2007}.

\begin{figure}[H]
  \begin{minipage}[b]{.5\linewidth}
     \centering
     \includegraphics[width=\textwidth]{re_HE_f.png}
     % plot 1
     \subcaption{Females}\label{fig:1.4a}
  \end{minipage}
  \hfill
  \begin{minipage}[b]{.5\linewidth}
     \centering
     \includegraphics[width=\textwidth]{re_HE_m.png}
     % plot 2
     \subcaption{Males}\label{fig:1.4b}
  \end{minipage}
  \caption{Rates of Returns to Higher and Vocational Education in Russia, RLMS 1994-2018}\label{fig:1.4}
\end{figure}

%%%%%%%%%%%%%%%%%%%%%%%%%%%%%%%%%%%%%%%%%%%%%%%%%%%%%%%%

\section{Depreciation of Human Capital \\ in the Russian Federation}

Age-earnings profiles are almost invariably concave downward shaped. Earnings rise after a labor market entrant completes full-time schooling. The profile indicates a peak in earnings, usually a few years before retirement, after which there is a steady decline in earnings. The concave shape is an outcome of two countervailing tendencies - the rise attributed to continued accumulation of human capital through training and the decline due to depreciation. The precise shape and location of the peak is an object of analytical interest. Depreciation of human capital is useful to investigate from a policy perspective. Just like some physical capital (machinery, buildings) are built stronger and last longer, is it possible that some kinds of education inherently generate human capital that is slower to depreciate? What attributes of the labor market lead to lower or higher levels of depreciation? What about the welfare implications of changes in the age at which individuals retire from the labor force? How has the depreciation rate of human capital changed over time in the Russian Federation? In particular, can changes in the rate of depreciation explain the observed pattern of the rates of return depicted in \ref{fig:1.2}?  

\subsection{Analytical Treatment of Depreciation} 

\citet{rosen_176._1976}  and \citet{mincer_175._1982} presented early treatments on the depreciation of human capital. However, in terms of a focus on depreciation, a seminal paper of \citet{neuman_091._1995} established the basic parameters that have guided the research since that time. The authors introduce the important distinction between two kinds of depreciation or loss of productive potential of human capital. The first one, termed as ``obsolescence" or ``vintage effect", is due to an overall upgrading of technology or the operation of other market forces that lowers the value of education or training obtained in a previous period. This is also termed as an `external depreciation", presumably as it is a given for an individual. The second kind of depreciation is attributed to the deterioration of physical and mental abilities of an individual due to the progression of a person's age, or the simple passage of time. This is also termed as ``internal depreciation". Neuman and Weiss posited that external effects would be more important for higher levels of education, under the assumption that changes in the labor market are transmitted more readily to higher education. They give the example that a recently educated electrical engineer would be learning many new things compared to one who studied the same subject in an earlier time. Neuman and Weiss reasoned that workers with basic education levels may not suffer as much from obsolescence. 

Figure \ref{fig:2.1} shows for the Russian Federation the effects described by Neuman and Weiss. There are three panels in the figure, and three lines in each figure. The vertical axis indicates the monthly earnings in constant 2018 rubles, using the Rosstat CPI deflator. The horizontal axis indicates the years of experience. The dotted line shows the earnings for 1998, the dashed line represents 2006 and the solid line the data from 2018. Each of the panels, representing a different level of education, shows an upward drift in the experience-earnings profiles in the period from 1998 to 2018. Only Figure \ref{fig:2.1a} shows a clear concave downwards profile for Higher Education; the concave tendency is less pronounced for the other two levels of Vocational education and Secondary education.
	
	\begin{figure}[H]
		\begin{minipage}[b]{.3\linewidth}
			\centering
			\hspace*{-0.7in}
			\includegraphics[width=150pt]{dp01_he.png}
			% plot 1
			\subcaption{Higher Education}\label{fig:2.1a}
		\end{minipage}
		\hfill
		\begin{minipage}[b]{.3\linewidth}
			\centering
			\hspace*{-0.7in}
			\includegraphics[width=150pt]{dp01_ve.png}
			% plot 2
			\subcaption{Vocational Education}\label{fig:2.1b}
		\end{minipage}
		\hfill
		\begin{minipage}[b]{.3\linewidth}
			\centering
			\hspace*{-0.7in}
			\includegraphics[width=150pt]{dp01_se.png}
			% plot 2
			\subcaption{Secondary Education}\label{fig:2.1c}
		\end{minipage}
		\caption{Neuman-Weiss vintage effects by education level from RLMS Rounds 1998, 2006 and 2018}\label{fig:2.1}
	\end{figure}
	
Putting the curves together by year (Figure \ref{fig:2.2})suggests that the premium for university education over the other two levels does narrow at higher levels of experience. In the figure, to accomodate the relatively lower wage levels of 1998, the leftmost panel (Figure \ref{fig:2.2a}) is slightly compressed compared to the other two panels. The  converging tendency between levels of education would suggest that depreciation is indeed higher for university graduates. In the next two subsections, we present a more rigorous quantitative treatment of this issue, using a variant of Neuman-Weiss developed by \citet{murillo_172._2006} and an alternative approach developed by \citet{arrazola_132b._2005}.

	
	\begin{figure}[H]
		\begin{minipage}[b]{.3\linewidth}
			\centering
			\hspace*{-0.7in}
			\includegraphics[width=150pt]{dp01_98.png}
			% plot 1
			\subcaption{1998}\label{fig:2.2a}
		\end{minipage}
		\hfill
		\begin{minipage}[b]{.3\linewidth}
			\centering
			\hspace*{-0.7in}
			\includegraphics[width=150pt]{dp01_06.png}
			% plot 2
			\subcaption{2006}\label{fig:2.2b}
		\end{minipage}
		\hfill
		\begin{minipage}[b]{.3\linewidth}
			\centering
			\hspace*{-0.7in}
			\includegraphics[width=150pt]{dp01_18.png}
			% plot 2
			\subcaption{2018}\label{fig:2.2c}
		\end{minipage}
		\caption{Neuman-Weiss vintage effects by Year from RLMS Rounds 1998, 2006 and 2018}\label{fig:2.2}
	\end{figure}

\subsection{Differential Depreciation Affecting Education and Training}
	
\citet{murillo_172._2006} implemented a variation of the Neuman and Weiss model with a focus on empirical implementation to Spain. We follow the Murillo notation in the implementation of the model, which begins with the following earnings equation: 
\begin{flalign}\label{eq:2.1} 
log(W_{T}) = \alpha + \beta_{1}KS_{T} +  \beta_{2}KE_{T} &&
\end{flalign}
\noindent
where $W$ represents earnings, $KS$ the stock of human capital derived from schooling of $S$ years, and $KE$ the stock of human capital acquired from on the job training or experience, and $T$ indexes the number of experience years since completing formal education. In this set-up, the parameters $\beta_1$ and $\beta_2$ are the productivity parameters for the respective parts of the stock of human capital. Both are assumed to suffer from depreciation or the loss of productive value. At this stage, we do not distinguish between the causes (internal or external) of this loss. The path of the stock of human capital due to education is given by 
\begin{flalign}\label{eq:2.2} 
KS_{T} = S + hTS &&
\end{flalign} 
\noindent
where $h$ is the rate of loss of the stock. The next equation for the loss of stock gained from experience is a bit more complicated. The stock from schooling, $S$ is taken to be fixed at the end of the full-time schooling period and the beginning of the working period. However, experience is being built up every year at the same time as the capital acquired from previous experience depreciates. 
\begin{flalign}\label{eq:2.3} 
KE_{T} = \{1 + (T-1)\cdot\gamma \} + \{1 + (T-2)\cdot\gamma \}  + \{1 + (T-3)\cdot\gamma \} + \ldots + \{1\}  && 
\end{flalign} 
\noindent
where $\gamma$ is the rate of loss applied every year. The equation can be simplified and summarized as
\begin{flalign}\label{eq:2.4} 
KE_{T} =  T + \gamma\cdot\{(T-1) + (T-2) + (T-3) + \ldots + 1\} = T + \gamma\cdot\frac{T^2}{2}   && 
\end{flalign} 
Substituting equations \ref{eq:2.2} and \ref{eq:2.4} into equation \ref{eq:2.1}, we get
\begin{flalign}\label{eq:2.5} 
log(W) = \alpha +  \beta_{1}S +  \beta_{1}hTS + \beta_{2}T + \frac{\beta_{2}\gamma}{2}T^{2} =  \alpha +  \beta_{1}S + \pi_{1}TS + \beta_{2}T + \pi_{2}T^{2} && 
\end{flalign} 
\noindent
where $\pi_{1} = \beta_{1}h$ and $\pi_{2} = \frac{\beta_{2}\gamma}{2}$. 
\vspace{2pt}
\noindent
From \ref{eq:2.5}, the depreciation rate during $T$ years applied to schooling can be computed as $\pi_{1}S $ and the depreciation rate applied to experience as $ 2\pi_{2}T$.
\vspace{-5pt}
\subsubsection{Estimation Results}
Table \ref{tab:2.1} shows OLS estimation results of equation \ref{eq:2.5} run on the whole sample of the RLMS observations. In this subsection, we analyzed separately six years that represent the ends (1994 and 2018), the diffused peak (2003 and 2006), and halfway points to the ends (2012 and 2003) of the available time interval. The idea is to examine the role played by changes in depreciation to explain the observed pattern of variation in the rates of return shown in Figure \ref{fig:1.1}. 

\setlength{\tabcolsep}{2pt}
\begin{table}[h!]
	\centering 
	\caption{Results of Estimating Human Capital Depreciation for the Whole Sample, RLMS} 
	\label{tab:2.1}
	\begin{tabular}{@{\extracolsep{3pt}}lcccccc} 
		\\[-1.8ex]\hline 
		\hline \\[-1.8ex] 
		& \textbf{1994} & \textbf{1998} & \textbf{2003} & \textbf{2006} & \textbf{2012} & \textbf{2018} \\ 
		\\[-1.8ex] & (1) & (2) & (3) & (4) & (5) & (6)\\ 
		\hline \\[-1.8ex] 
		Constant & 10.266$^{***}$ & 4.720$^{***}$ & 6.762$^{***}$ & 7.854$^{***}$ & 8.889$^{***}$ & 9.205$^{***}$ \\ 
		& (0.301) & (0.258) & (0.221) & (0.181) & (0.128) & (0.158) \\ 
		& & & & & & \\ 
		Educ, years ($S$) & 0.113$^{***}$ & 0.116$^{***}$ & 0.094$^{***}$ & 0.074$^{***}$ & 0.054$^{***}$ & 0.053$^{***}$ \\ 
		& (0.020) & (0.017) & (0.015) & (0.012) & (0.008) & (0.010) \\ 
		& & & & & & \\ 
		Educ X Exper ($TS$) & $-$0.001$^{*}$ & $-$0.001$^{*}$ & $-$0.00005 & 0.0003 & 0.0003 & 0.0001 \\ 
		& (0.001) & (0.001) & (0.001) & (0.0005) & (0.0003) & (0.0004) \\ 
		& & & & & & \\ 
		Exper($T$) & 0.053$^{***}$ & 0.044$^{***}$ & 0.016 & $-$0.001 & 0.012$^{*}$ & 0.023$^{***}$ \\ 
		& (0.015) & (0.013) & (0.011) & (0.009) & (0.007) & (0.008) \\ 
		& & & & & & \\ 
		Exper squared ($T^2$) & $-$0.001$^{***}$ & $-$0.001$^{***}$ & $-$0.0004$^{***}$ & $-$0.0002$^{*}$ & $-$0.001$^{***}$ & $-$0.001$^{***}$ \\ 
		& (0.0002) & (0.0001) & (0.0001) & (0.0001) & (0.0001) & (0.0001) \\ 
		& & & & & & \\ 
		\hline \\[-1.8ex] 
		Observations & 3,037 & 3,100 & 3,856 & 4,800 & 7,417 & 6,112 \\ 
		R$^{2}$ & 0.043 & 0.058 & 0.068 & 0.078 & 0.088 & 0.071 \\ 
		Adjusted R$^{2}$ & 0.042 & 0.057 & 0.067 & 0.077 & 0.087 & 0.071 \\ 
		Residual Std. Error & 0.934 & 0.800 & 0.782 & 0.715 & 0.666 & 0.617 \\ 
		F Statistic & 34.062$^{***}$ & 47.678$^{***}$ & 69.846$^{***}$ & 101.053$^{***}$ & 177.952$^{***}$ & 117.104$^{***}$ \\ 
		\hline 
		\hline \\[-1.8ex] 
		\textit{Note:}  & \multicolumn{6}{r}{$^{*}$p$<$0.1; $^{**}$p$<$0.05; $^{***}$p$<$0.01} \\ 
	\end{tabular} 
\end{table}


Using the coefficient estimates derived from Table \ref{tab:2.1}, we compute the depreciation rate during $T$ years applied to schooling as $\pi_{1}S $ and the depreciation rate applied to experience as $ 2\pi_{2}T$, evaluating the expression at the mean level of schooling. Table \ref{tab:2.2} reports the depreciation rate values so calculated with the corresponding sample means. The table shows an interesting U-shaped pattern in the depreciation rate for human capital, attributable mainly to the depreciation rate associated with experience. The depreciation rate associated with education has been declining steadily and did not pick up again as measured with the given data. The depreciation rate associated with experience declined at first and then picked up again. 

\begin{center}
\captionof{table}{Average Depreciation Rate by Years}\label{tab:2.2}
\keepXColumns
\begin{tabularx}{\textwidth}{@{\extracolsep{5pt}}rlrrrrrrc}
		\hline
\multicolumn{9}{l}{\textbf{Panel A: Whole Sample}} \\
\hline
			& \textbf{Statistic} & \textbf{1994} & \textbf{1998} & \textbf{2003} & \textbf{2006} & \textbf{2012} & \textbf{2018} &  \\ 
		\hline
		1 & Experience, mean   & 21.41 & 22.32 & 22.20 & 22.24 & 22.52 & 22.52 & \\
		2 & Education, mean & 12.70 & 12.69 & 12.79 & 12.79 & 12.95 & 13.27 &\\
		\midrule
		3 & DR Experience, \% & 1.87 & 1.55 & 1.04 & 0.50 & 1.37 & 1.63 & 
\graph{1}{1}{C:/Country/Russia/Data/SEASHELL/SEABYTE/Edreru/wp1/sparklines/all2-1} \\ 
		4 & DR Education, \% & 2.80 & 2.71 & 0.11 & 0.00 & 0.00 & 0.00 &
\graph{1}{1}{C:/Country/Russia/Data/SEASHELL/SEABYTE/Edreru/wp1/sparklines/all2-2} \\ 
		5 & DR Human Capital, \% & 4.67 & 4.26 & 1.15 & 0.50 & 1.37 & 1.63 & 
\graph{1}{1}{C:/Country/Russia/Data/SEASHELL/SEABYTE/Edreru/wp1/sparklines/all2-3}\\ 
		\hline
\end{tabularx}
%\par\noindent
\begin{tabularx}{\textwidth}{@{\extracolsep{5pt}}rlrrrrrrc}
		\hline
\multicolumn{9}{l}{\textbf{Panel B: Female Sample}} \\
		\hline
  1 & Experience, mean & 21.36 & 22.09 & 22.34 & 22.33 & 22.69 & 22.67 & \\  
  2 & Education, mean & 12.76 & 12.85 & 12.98 & 13.05 & 13.24 & 13.58 & \\ 
  3 & DR Experience, \% & 2.46 & 2.57 & 1.62 & 0.78 & 1.23 & 1.52 & 
\graph{1}{1}{C:/Country/Russia/Data/SEASHELL/SEABYTE/Edreru/wp1/sparklines/female2-1} \\  
  4 & DR Education, \% & 3.81 & 5.31 & 3.97 & 0.00 & 0.00 & 0.00 & 
\graph{1}{1}{C:/Country/Russia/Data/SEASHELL/SEABYTE/Edreru/wp1/sparklines/female2-2} \\
  5 & DR Human Capital, \% & 6.27 & 7.88 & 5.59 & 0.78 & 1.23 & 1.52 & 
\graph{1}{1}{C:/Country/Russia/Data/SEASHELL/SEABYTE/Edreru/wp1/sparklines/female2-3} \\ 
   \hline
\end{tabularx}
\begin{tabularx}{\textwidth}{@{\extracolsep{5pt}}rlrrrrrrc}
		\hline
\multicolumn{9}{l}{\textbf{Panel C: Male Sample}} \\
  \hline
1 & Experience, mean & 21.47 & 22.58 & 22.02 & 22.14 & 22.31 & 22.34 & \\ 
  2 & Education, mean & 12.62 & 12.50 & 12.57 & 12.47 & 12.61 & 12.91 & \\  
  3 & DR Experience, \% & 1.83 & 1.08 & 0.80 & 0.67 & 2.23 & 1.91 & 
\graph{1}{1}{C:/Country/Russia/Data/SEASHELL/SEABYTE/Edreru/wp1/sparklines/male2-1} 
\\ 
  4 & DR Education, \% & 3.96 & 2.74 & 0.91 & 0.00 & 0.00 & 0.00 & 
\graph{1}{1}{C:/Country/Russia/Data/SEASHELL/SEABYTE/Edreru/wp1/sparklines/male2-2} 
\\ 
  5 & DR Human Capital, \% & 5.78 & 3.82 & 1.71 & 0.67 & 2.23 & 1.91 &
\graph{1}{1}{C:/Country/Russia/Data/SEASHELL/SEABYTE/Edreru/wp1/sparklines/male2-3} 
\\ 
   \hline
\end{tabularx}
\end{center}

\setcounter{table}{2} % to avoid tabularx counting subtables as tables

Further work is required, including computation of the depreciation rates at levels other than the mean values. At this stage, the findings raise some interesting questions which needs to be addressed by further research. In the period from 1994 to 2006, the depreciation rate appears to be declining, just as the rates of return were on an ascending curve. As both kinds of depreciation (for experience and education) were declining, it is possible that the main cause was in the labor market experience rather than in the education system. Since the peak of earnings premiums in the 2003-2006 period, as returns to education have declined, we see that the depreciation rates associated with experience have started climbing back, but depreciation rates associated with education have declined to null and not reverted. It is tempting to claim that this indicates a qualitative improvment in the skills provided by the education system, but further investigation is warranted before making such a claim. We explore next an alternative computation of the depreciation rate. 

\subsection{Depreciation of Human Capital using Non-Linear Least Squares}

\citet{arrazola_132b._2005} developed an alternative approach the issue of human capital depreciation with a first principles approach regarding the formation of human capital, providing an empirical estimation for Spain.  A number of other authors have replicated Arrazola's approach. In this paper, we follow the notation adopted by Sylvain Weber, who estimated depreciation rates for Switzerland (\citet{weber_173._2008} and \citet{weber_156._2011}). Weber starts with the definition of $s_{t}$ -- the time fraction invested into the generation of new human capital by a person at age $t$. Relying on a human capital theory implication about the decline of $s_{t}$ over the life cycle, Weber shows that the complete path of $s_{t}$ is written as follows:

\begin{flalign}\label{eq:2.6} 
s_{t}=\left\{\begin{array}{ll}
{0} & {\text { if } t<6} \\
{1} & {\text { if } 6 \leq t<S^{\star}} \\
{\alpha-\frac{\mathlarger{\alpha}}{T-S^{*}} \cdot\left(t-S^{\star}\right)=\alpha \cdot\left(1-\frac{X_{t}}{L}\right)} & {\text { if } S^{\star} \leq t \leq T}
\end{array}\right.
\end{flalign}

\noindent
where $\alpha$ is a parameter, $S^{*}$ is the age when schooling life ends and the working one begins, $T$ is the retirement age, $L = T - S^{*}$ is the total working life length, $X_{t} = t - S^{*}$ is experience. Schooling duration is equal to $S^{*} - 6$.

The model then utilizes the standard human capital theory specification that potential earnings $E_{t}$ are exponentially related to the human capital stock:

\begin{flalign}\label{eq:2.7} 
E_{t}=W \cdot \exp \left(\beta_{K} K_{t}+\beta_{Z} Z_{t}\right)
\end{flalign}

\noindent
where $W$ is a return per period on a unit of earnings capacity, $K_{t}$  is the stock of human capital at time $t$, $Z_{t}$ is a set of observable attributes supposed to influence on earnings, and $K, Z$ are the parameters of interest. The stock of human capital in period $t$ can be estimated as the sum of the stock from the previous period minus the loss due to depreciation plus the quantity generated during the $t_{th}$ period:

\begin{flalign}\label{eq:2.8} 
K_{t}=K_{t-1}-\delta \cdot K_{t-1}+\Delta K_{t}=(1-\delta) \cdot K_{t-1}+\Delta K_{t}
\end{flalign}

By recursion, an expression for $K_{t}$ as a function of the human capital stock acquired at the end of formal education $K_{S}$ is given by:

\begin{flalign}\label{eq:2.9}
K_{t}=(1-\delta)^{t} \cdot K_{S}+\sum_{j=S^{*}}^{t-1}(1-\delta)^{j} \cdot \Delta K_{t-j}
\end{flalign}

Taking the logarithms of the expression \ref{eq:2.7} and substituting $K_{t}$ by the equation \ref{eq:2.9} leads to:

\begin{flalign}\label{eq:2.10} 
\ln E_{t}=\ln W+\beta_{K} \cdot\left\{(1-\delta)^{t} \cdot K_{S}+\sum_{j=S^{*}}^{t-1}(1-\delta)^{j} \cdot \Delta K_{t-j}\right\}+\beta_{Z} Z_{t}
\end{flalign}

Next is the standard human capital relationship between observed and potential earnings. As only a proportion of $s_{t}$  of the human capital stock is  used in the actual production of earnings, observed earnings can be expressed by:

\begin{flalign}\label{eq:2.11} 
\begin{aligned}
&Y_{t}=\left(1-s_{t}\right) \cdot E_{t}\\
&\ln Y_{t}=\ln \left(1-s_{t}\right)+\ln E_{t}
\end{aligned}
\end{flalign}

Combining \ref{eq:2.10} and \ref{eq:2.11} results in:

\begin{flalign}\label{eq:2.12} 
\ln Y_{t}=\ln W+\beta_{K} \cdot\left\{(1-\delta)^{t} \cdot K_{S}+\sum_{j=S^{*}}^{t-1}(1-\delta)^{j} \cdot \Delta K_{t-j}\right\} + \beta_{Z} Z_{t} + \ln \left(1-s_{t}\right)
\end{flalign}

Finally, as the human capital stock at the end of education is related to the human capital received, there is a direct association between this stock and the schooling duration:

\begin{flalign}\label{eq:2.13} 
K_{S}=S
\end{flalign}

The production of new human capital $K_{t}$ depends on the portion of time devoted to this activity:

\begin{flalign}\label{eq:2.14} 
\Delta K_{t}=s_{t}=\left\{\begin{array}{ll}
{0} & {\text { if } t<6} \\
{1} & {\text { if } 6 \leq t<S^{\star}} \\
{\alpha \cdot\left(1-\frac{X_{t}}{L}\right)} & {\text { if } S^{*} \leq t \leq T}
\end{array}\right.
\end{flalign}

Using \ref{eq:2.8} and \ref{eq:2.11} to express $K_{S}$ as a sum of the human capital quantities produced during schooling, the result is:

\begin{flalign}\label{eq:2.15} 
K_{S} \stackrel{(3)}{=} \sum_{j=0}^{S^{*}}(1-\delta)^{j} \cdot \Delta K_{S^{*}-j} \stackrel{(9)}{=} \sum_{j=6}^{S^{*}}(1-\delta)^{j}
\end{flalign}

Substituting \ref{eq:2.13} and \ref{eq:2.14} into \ref{eq:2.12}, adding an error term and an individual subscript $i$  provides the equation that can be estimated using non-linear least squares (NLS):

\begin{multline}\label{eq:2.16} 
\ln Y_{i t}= \ln W+\beta_{K} \cdot\left\{(1-\delta)^{X_{i t}} \cdot S_{i}+\alpha \cdot \frac{1-(1-\delta)^{X_{i t}}}{\delta}\right.\cdot\\
\left.\left(1+\frac{1-\delta}{\delta \cdot L_{i}}\right)-\frac{\alpha \cdot X_{i t}}{\delta \cdot L_{i}}\right\}+\ln \left\{1-\left(\alpha-\frac{\alpha}{L_{i}} \cdot X_{i t}\right)\right\}+\beta_{Z} \cdot Z_{i t}+u_{i t}
\end{multline}

\noindent
where $t$ shows a time period, $\ln Y$ is a logarithm of the observed earnings, $\ln W$ is a logarithm of a return per certain period on a unit of earnings capacity, $\beta_{K}$ is the effect of the human capital stock on earnings, $\beta_{Z}$ is the effect of other covariates in the model on earning, $\delta$ is the human capital depreciation rate, $X_{i t}$ is the labor market experience, $L_{i}$ is the total working life length, $\alpha$ is a parameter reflecting the share of time invested in training, $Z_{i t}$ is a set of observable attributes hypothesized to have an impact on earnings, $u_{i t}$ is an error term.

Table \ref{tab:2.3} reports empirical findings for the estimation of the \ref{eq:2.16} equation using NLS with robust standard errors for the same range of years as presented in the previous section. Unlike that earlier model, the Arrazola model does not allow for a different treatment of depreciation of human capital acquired from schooling or from experience - only a single $\delta$ (depreciation rate of the human capital) parameter is estimated. However, the model does allow the identification of an  $\alpha$ parameter (related to post-school investment in human capital).

\begin{center}
	\captionof{table}{Non-Linear Lest Squares estimated for range of years}
	\label{tab:2.3}
	\keepXColumns
	\begin{tabularx}{\textwidth}{clccccccc}
		\hline
		\multicolumn{9}{l}{\textbf{Panel A: Whole Sample}} \\
		\hline
		& Parameter & 1994 & 1998 & 2003 & 2006 & 2012 & 2018 & \\ 
		\hline
		 & lnW & 10.4780 & 4.8622 & 6.7305 & 7.8405 & 8.4104 & 8.8524 & \\ 
		 &  & (0.1913) & (0.1646) & (0.1409) & (0.0838) & (0.0787) & (0.0885) & \\ 
		 & bk & 0.1453 & 0.1429 & 0.1144 & 0.0723 & 0.1382 & 0.1487 & \\ 
		 &  & (0.0167) & (0.0144) & (0.0140) & (0.0106) & (0.0087) & (0.0086) & \\ 
		 & delta & 0.0246 & 0.0208 & 0.0093 & -0.0040 & 0.0369 & 0.0459 & 
		\graph{1}{1}{C:/Country/Russia/Data/SEASHELL/SEABYTE/Edreru/wp1/sparklines/Weber_sprk_all2-1}\\ 
		 &  & (0.0052) & (0.0043) & (0.0050) & (0.0058) & (0.0043) & (0.0051) & \\ 
		 & alpha & 0.4798 & 0.3860 & 0.1352 & -0.1690 & 0.4972 & 0.6686 & 
		\graph{1}{1}{C:/Country/Russia/Data/SEASHELL/SEABYTE/Edreru/wp1/sparklines/Weber_sprk_all2-2}\\ 
		 &  & (0.0912) & (0.0790) & (0.0911) & (0.0950) & (0.0601) & (0.0533) & \\ 
		 & Sample size & 3037 & 3100 & 3856 & 4800 & 7417 & 6112 & \\ 
		\hline
\end{tabularx}

\begin{tabularx}{\textwidth}{clccccccc}
		\hline
		\multicolumn{9}{l}{\textbf{Panel B: Female Sample}} \\
		\hline
		& Parameter & 1994 & 1998 & 2003 & 2006 & 2012 & 2018 & \\ 
		\hline
		 & lnW & 10.1580 & 4.1353 & 5.7238 & 6.9251 & 7.9143 & 8.4131 & \\ 
		 &  & (0.2447) & (0.2124) & (0.1973) & (0.1663) & (0.1136) & (0.1275) & \\ 
		 & bk & 0.1524 & 0.1818 & 0.1702 & 0.1321 & 0.1329 & 0.1330 & \\ 
		 &  & (0.0196) & (0.0163) & (0.0158) & (0.0149) & (0.0104) & (0.0103) & \\ 
		 & delta & 0.0275 & 0.0260 & 0.0156 & 0.0065 & 0.0197 & 0.0249 & 
		\graph{1}{1}{C:/Country/Russia/Data/SEASHELL/SEABYTE/Edreru/wp1/sparklines/Weber_sprk_f2-1}\\ 
		 &  & (0.0060) & (0.0042) & (0.0038) & (0.0044) & (0.0036) & (0.0036) & \\ 
		 & alpha & 0.5889 & 0.5408 & 0.3466 & 0.0900 & 0.3354 & 0.4628 & 
		\graph{1}{1}{C:/Country/Russia/Data/SEASHELL/SEABYTE/Edreru/wp1/sparklines/Weber_sprk_f2-2}\\ 
		 &  & (0.0974) & (0.0749) & (0.0763) & (0.0862) & (0.0659) & (0.0609) & \\ 
		 & Sample size & 1645 & 1667 & 2093 & 2630 & 4057 & 3312 & \\ 
		\hline	
\end{tabularx}

\begin{tabularx}{\textwidth}{clccccccc}
		\hline
		\multicolumn{9}{l}{\textbf{Panel C: Male Sample}} \\
		\hline
		& Parameter & 1994 & 1998 & 2003 & 2006 & 2012 & 2018 & \\ 
		\hline
		 & lnW & 10.4992 & 5.1267 & 7.3195 & 8.1556 & 8.2117 & 8.8384 & \\ 
		 &  & (0.2880) & (0.2420) & (0.1530) & (0.1158) & (0.1195) & (0.1213) & \\ 
		 & bk & 0.1697 & 0.1425 & 0.0845 & 0.0725 & 0.2206 & 0.1784 & \\ 
		 &  & (0.0244) & (0.0215) & (0.0180) & (0.0163) & (0.0111) & (0.0118) & \\ 
		 & delta & 0.0261 & 0.0168 & -0.0020 & 0.0015 & 0.0595 & 0.0511 & 
		\graph{1}{1}{C:/Country/Russia/Data/SEASHELL/SEABYTE/Edreru/wp1/sparklines/Weber_sprk_m2-1}\\ 
		 &  & (0.0067) & (0.0059) & (0.0082) & (0.0095) & (0.0063) & (0.0069) & \\ 
		 & alpha & 0.4625 & 0.2669 & -0.1351 & -0.1196 & 0.8161 & 0.7312 & 
		\graph{1}{1}{C:/Country/Russia/Data/SEASHELL/SEABYTE/Edreru/wp1/sparklines/Weber_sprk_m2-2}\\ 
		 &  & (0.1278) & (0.1162) & (0.1362) & (0.1475) & (0.0484) & (0.0663) & \\ 
		 & Sample size & 1392 & 1433 & 1763 & 2170 & 3360 & 2800 & \\ 
		\hline	
\end{tabularx}
\end{center}

\setcounter{table}{3} % to avoid tabularx countine subtables as tables

The sparklines in Table \ref{tab:2.3} indicates a smilar roughly U-shaped  pattern for depreciation as reported in Table \ref{tab:2.2}, with depreciation of human capital first declining and then increasing again. This supports the narrative that the observed increase and then decrease in returns to education in the Russian Federation may be explained through the effect of depreciation. The  exact magnitudes of estimated depreciation in the two tables do not match - while the range of depreciation is similar - between 2\% to 5\%, the 2018 figures indicate a higher level in Table \ref{tab:2.3}. 

\vspace{3pt}

An intriguing finding concerns the difference in depreciation rates between female and male workers. The conventional human capital logic holds that women typically face longer periods outside of the labor market because of child-bearing and child-rearing responsibilities. Absence from the labor market would lead to higher levels of depreciation amongst women. In the case of the Russian Federation, the estimates of both Table \ref{tab:2.2} and Table \ref{tab:2.3} reflect this pattern in the first half of the period, up until the estimates for 2006. Around the time of the peak in returns, the depreciation rate drops to zero for both men and women, but in the subsequent period, the depreciation rate for men appears to be higher than the rate for women. The fact that both methodologies reflect this pattern indicates a real phenomenon, rather than a statistical artefact, and something to be explored further. 

\vspace{3pt}

Finally, a word about the $\alpha$ parameter, which is an indicator of post-schooling investment in human capital. This parameter also shows a similar tendency as the depreciation rate, meaning a decline to zero and a subsequent  increase. As with depreciation, the first half shows a higher $\alpha$ for female workers until it drops to zero for both males and females at the time of peak returns, and in the subsequent period the $\alpha$ parameter level is higher for males. 

\vspace{3pt}

Adapting a strategy adopted by (\citet{weber_173._2008} to fit the Russian context, Table \ref{tab:2.4} provides four alternative specifications displayed separately by gender. The four models portray the following combinations regarding the $\alpha$  and $\delta$ parameters: \textit{model I} - both $\alpha$ and $\delta$ are constant across education levels; \textit{model II} - $\alpha$ is constant, $\delta$ varies; \textit{model III} - $\alpha$ varies, $\delta$ is constant, \textit{model VI} - both $\alpha$ and $\delta$ vary across education levels.

\vspace{3pt}

Model I - the base model, has already been presented in Table \ref{tab:2.3} and is shown again as part of Table \ref{tab:2.4} only for easy reference. Model II, allows the $\delta$ parameter to vary across education levels; Model III allows the $\alpha$ parameter to vary across education levels; and finally Model IV allows both parameters to vary by education level. The estimates indicate the absence of depreciation effects by educational level. Weber had found for Switzerland  that depreciation is higher for vocational education, and provided the explanation that vocational education skills tend to be more specific to jobs and careers. However, this finding is not replicated with the data for the Russian Federation. The statistically signifciant finding in Table 2.4 concerns the $\alpha$ parameter. Post-schooling investment in human capital for those with vocational education is not different from those with secondary education, but university education brings with it a higher level of the $\alpha$ parameter, for both male and female workers. 

\setlength{\tabcolsep}{2pt}
\begin{table}[H]\centering
	\def\sym#1{\ifmmode^{#1}\else\(^{#1}\)\fi}
	\caption{Empirical Estimates for Females and Males, RLMS 2018}
	\label{tab:2.4}
	\begin{tabular}{l*{8}{c}}
		\toprule\\
		&\multicolumn{4}{c}{Females}&\multicolumn{4}{c}{Males}\\
		\cmidrule(lr){2-5}\cmidrule(lr){6-9}
		&\multicolumn{1}{c}{I}&\multicolumn{1}{c}{II}&\multicolumn{1}{c}{III}&\multicolumn{1}{c}{IV}&\multicolumn{1}{c}{I}&\multicolumn{1}{c}{II}&\multicolumn{1}{c}{III}&\multicolumn{1}{c}{IV}\\
		\cmidrule(lr){1-5}\cmidrule(lr){6-9}
		lnW&8.413\sym{***}&8.901\sym{***}&8.778\sym{***}&8.644\sym{***}&8.838\sym{***}&8.950\sym{***}&9.022\sym{***}&8.864\sym{***}\\
		&(0.127)&(0.319)&(0.0975)&(0.279)&(0.121)&(0.291)&(0.0925)&(0.221)\\
		\hline
		bk&0.133\sym{***}&0.111\sym{***}&0.125\sym{***}&0.129\sym{***}&0.178\sym{***}&0.177\sym{***}&0.183\sym{***}&0.179\sym{***}\\
		&(0.0103)&(0.0115)&(0.0105)&(0.0130)&(0.0118)&(0.0147)&(0.0122)&(0.0111)\\
		\hline
		delta&0.0249\sym{***}&&&&0.0511\sym{***}&&&\\
		&(0.00357)&&&&(0.00692)&&&\\
		\hline
		alpha&0.463\sym{***}&0.553\sym{***}&&&0.731\sym{***}&0.761\sym{***}&&\\
		&(0.0609)&(0.143)&&&(0.0663)&(0.124)&&\\
		\hline
		delta\_base&&0.0387\sym{*}&0.0355\sym{***}&0.0305\sym{*}&&0.0558\sym{**}&0.0597\sym{***}&0.0431\sym{**}\\
		&&(0.0181)&(0.00453)&(0.0134)&&(0.0185)&(0.00506)&(0.0144)\\
		\hline
		delta\_voc&&-0.00109&&-0.00128&&0.000258&&0.00694\\
		&&(0.00251)&&(0.00519)&&(0.00150)&&(0.00547)\\
		\hline
		delta\_uni&&-0.00699&&0.00198&&-0.00143&&0.00892\\
		&&(0.00610)&&(0.00805)&&(0.00333)&&(0.00728)\\
		\hline
		alpha\_base&&&0.448\sym{***}&0.424\sym{*}&&&0.698\sym{***}&0.498\sym{**}\\
		&&&(0.0728)&(0.172)&&&(0.0578)&(0.166)\\
		\hline
		alpha\_voc&&&0.0124&-0.0322&&&0.00828&0.144\\
		&&&(0.0442)&(0.128)&&&(0.0381)&(0.114)\\
		\hline
		alpha\_uni&&&0.208\sym{***}&0.206&&&0.157\sym{***}&0.307\sym{*}\\
		&&&(0.0533)&(0.134)&&&(0.0420)&(0.125)\\
		\hline
		\(N\)&3312&3312&3312&3312&2800&2800&2800&2800\\
		adj.\(R^{2}\)&0.082&0.085&0.091&0.091&0.096&0.096&0.106&0.107\\
		\textit{AIC}&6017.6&6006.2&5983.5&5987.1&4842.8&4844.9&4813.7&4814.6\\
		\textit{BIC}&6042.0&6042.8&6020.2&6035.9&4866.5&4880.5&4849.3&4862.1\\
		\bottomrule
		\multicolumn{9}{l}{\footnotesize Standard errors in parentheses}\\
		\multicolumn{9}{l}{\footnotesize \sym{*} \(p<0.05\), \sym{**} \(p<0.01\), \sym{***} \(p<0.001\)}\\
	\end{tabular}
\end{table}


\section{Further Exploration of Depreciation}
\subsection{Depreciation and the Gender dimension}

The previous section indicated an intriguing finding regarding depreciation and gender, with recent rounds of RLMS data indicating a higher level of depreciation for male workers, a possibly anamolous finding. This sub-section draws upon the literature regarding occupational gender segregation, the tendency for some jobs to be dominated by one gender, a phenomenon that has been studied in detail in labor markets around the world (\citet{preston_179_1999} and \citet{blau_178_2013}). Empirical application in the Russian Federation has examined trends regarding gender segregation in occupations (\citet{klimova_159._2009},\citet{klimova_131._2012},\citet{kosyakova_121._2015}). The bulk of this literature is concerned with the issue of the possible inequities and inefficiencies arising from gender segregation. However, for purpose of this paper is to exploit the presence in the data of occupational gender segregation to obtain insights into human capital depreciation.


In this section,we extend the Neuman and Weiss model of the previous section to compare the estimated depreciation rates between female- and male-dominated groups in various industries and occupations. The RLMS 2018 database contains information about sector or industry and standardized ISCO-08 classfication of jobs. These were used  to tag gender-based industrial sectors and occupations, respectively. The average female representation by sector was 54\%.  Using a range of $\pm$ 10\% from the average, a sector was marked as female-dominated if it contained more than 64\% of women workers, and male-dominated if it contained less than 44\% of women workers. Neutral sectors occupied the middle of the distribution. Figure \ref{fig:3.1} visualizes this procedure; Table \ref{tab:3.1} maps title of sectors with female percentages in them. To generate gender-related occupations a similar tactic was applied based on the 2-digit ISCO-08 classification (see Figure \ref{fig:3.2} and Table \ref{tab:3.2}). 

\begin{figure}[H]
	\centering
	\includegraphics{gen_ind18.png}
	\caption{Distribution of Employment in RLMS 2018 by Industry and Gender}\label{fig:3.1}
\end{figure}

\begin{figure}[H]
	\centering
	\includegraphics{gen_occ18.png}
	\caption{Distribution of Employment in RLMS 2018 by Occupation and Gender}\label{fig:3.2}
\end{figure}
	
\begin{table}[H]
		\centering
		\def\arraystretch{1} 
		\caption{Industries by Strength of Female Proportion, RLMS 2018}
		\label{tab:3.1}
		\begin{tabular}{p{2.5cm}l>{\raggedleft\arraybackslash}p{1.5cm}>{\raggedleft\arraybackslash}p{3cm}>{\raggedleft\arraybackslash}p{1.5cm}}
			\hline \hline
			\textbf{Category} & \textbf{Sector} & \textbf{N fem} & \textbf{\% fem} & \textbf{N total} \\ 
			\hline
			Female  & Social Services & 37 & 92.5\% &  40 \\ 
			dominated  & Other & 17 & 89.5\% &  19 \\ 
			& Education & 609 & 88.0\% & 692 \\ 
			& Public Health & 412 & 85.7\% & 481 \\ 
			& Real Estate Operations & 19 & 79.2\% &  24 \\ 
			& Government and Public Administration & 155 & 78.7\% & 197 \\ 
			& General Public Services & 15 & 75.0\% &  20 \\ 
			& Finance & 107 & 73.8\% & 145 \\ 
			& Science, Culture & 100 & 70.4\% & 142 \\ 
			& Jurisprudence & 19 & 67.9\% &  28 \\ \hline
			Neutral & Mass Media, Telecommunications & 24 & 63.2\% &  38 \\ 
			& Trade, Consumer Services & 738 & 62.8\% & 1175 \\ 
			& Light industry, Food industry & 209 & 55.0\% & 380 \\ 
			& Sports, Tourism,Entertainment & 18 & 54.5\% &  33 \\ \hline
			Male & Miltary Industrial Complex & 67 & 41.1\% & 163 \\ 
			dominated & Housing and Community Services & 95 & 39.1\% & 243 \\ 
			& Chemical Industry & 14 & 38.9\% &  36 \\ 
			& Civil Machine Construction & 51 & 37.8\% & 135 \\ 
			& Agriculture & 79 & 33.9\% & 233 \\ 
			& Transportation, Communication & 186 & 33.6\% & 553 \\ 
			& Information Technology & 9 & 32.1\% &  28 \\ 
			& Energy or Power Industry & 41 & 31.3\% & 131 \\ 
			& Army, Internal Security & 90 & 30.1\% & 299 \\ 
			& Other Heavy Industry & 60 & 28.7\% & 209 \\ 
			& Oil and Gas Industry & 52 & 23.5\% & 221 \\ 
			& Wood, Timber, Forestry & 7 & 21.2\% &  33 \\ 
			& Construction & 73 & 18.7\% & 391 \\ \hline
			& Total & 3303 &54.3\% & 6089 \\ \hline
		\end{tabular}
\end{table}
	
\begin{table}[!ht]
	\centering
	\caption{Occupations by Strength of Female Proportion, RLMS 2018}
	\label{tab:3.2}
		\begin{small}
			\begin{tabular}{cp{10cm}ccc}
				\hline
				& \textbf{Occupation} & \textbf{N fem} & \textbf{\% fem} & \textbf{N total} \\ 
				\hline
				1 & Personal Care Workers & 97 & 97.0\% & 100 \\ 
				2 & Cleaners and Helpers & 163 & 95.9\% & 170 \\ 
				3 & Food Preparation Assistants & 21 & 95.5\% &  22 \\ 
				4 & Teaching Professionals & 370 & 93.4\% & 396 \\ 
				5 & General and Keyboard Clerks & 71 & 93.4\% &  76 \\ 
				6 & Other Clerical Support Workers & 25 & 92.6\% &  27 \\ 
				7 & Business and Administration Professionals & 97 & 91.5\% & 106 \\ 
				8 & Health Associate Professionals & 192 & 91.4\% & 210 \\ 
				9 & Health Professionals & 79 & 87.8\% &  90 \\ 
				10 & Sales Workers & 350 & 86.6\% & 404 \\ 
				11 & Customer Services Clerks & 67 & 85.9\% &  78 \\ 
				12 & Legal, Social and Cultural Professionals & 169 & 80.5\% & 210 \\ 
				13 & Business and Administration Associate Professionals & 517 & 77.4\% & 668 \\ 
				14 & Food Processing, Woodworking, Garment and Other Craft and Related Trades Workers & 51 & 77.3\% &  66 \\ 
				15 & Administrative and Commercial Managers & 25 & 71.4\% &  35 \\ 
				16 & Personal Services Workers & 172 & 70.5\% & 244 \\ 
				17 & Hospitality, Retail and Other Services Managers & 38 & 69.1\% &  55 \\ 
				18 & Legal, Social, Cultural and Related Associate Professionals & 69 & 65.7\% & 105 \\ \hline
				19 & Numerical and Material Recording Clerks & 100 & 63.3\% & 158 \\ 
				20 & Production and Specialized Services Managers & 139 & 46.0\% & 302 \\ 
				21 & Handicraft and Printing Workers & 9 & 45.0\% &  20 \\ \hline
				22 & Science and Engineering Professionals & 101 & 42.8\% & 236 \\ 
				23 & Stationary Plant and Machine Operators & 72 & 39.6\% & 182 \\ 
				24 & Agricultural, Forestry and Fishery Labourers & 11 & 39.3\% &  28 \\ 
				25 & Refuse Workers and Other Elementary Workers & 30 & 39.0\% &  77 \\ 
				26 & Miscellaneous non-ISCO & 9 & 36.0\% &  25 \\ 
				27 & Science and Engineering Associate Professionals & 120 & 34.9\% & 344 \\ 
				28 & Chief Executives, Senior Officials and Legislators & 7 & 29.2\% &  24 \\ 
				29 & Information and Communications Technology Professionals & 15 & 21.7\% &  69 \\ 
				30 & Labourers in Mining, Construction, Manufacturing and Transport & 24 & 20.9\% & 115 \\ 
				31 & Assemblers & 11 & 18.3\% &  60 \\ 
				32 & Building and Related Trades Workers (excluding Electricians) & 23 & 11.1\% & 207 \\ 
				33 & Protective Services Workers & 23 & 10.7\% & 215 \\ 
				34 & Electrical and Electronic Trades Workers & 16 & 9.9\% & 162 \\ 
				35 & Drivers and Mobile Plant Operators & 23 & 4.1\% & 558 \\ 
				36 & Metal, Machinery and Related Trades Workers & 6 & 2.2\% & 267 \\ 
				\hline
			\end{tabular}
		\end{small}
	\end{table}

The Neuman and Weiss model provides an estimation of the depreciation rate for human capital, but by itself is unable to identify how much of that  depreciation is external or internal. External depreciation is due to obsolescence (as new technologies make skills redundant) and internal depreciation is due to factors related to the individual. The previous section reported the finding that depreciation for female workers first exceeded and  then was exceeded by depreciation for male workers. Examining differences in depreciation rate by the segregation classfication helps identify between internal and external depreciation based on a conjecture. The conjecture is that external depreciation would have a greater affect by industry sector, as technological change would propagate more rapidly through a sector rather than through occupations, which are dispersed across sectors. 

Table \ref{tab:3.3} depicts average rates of human capital loss due to experience and education by the female- and male-dominated industrial sectors and occupations. Industry or sector related differences does show difference in the depreciation rate, with depreciation rate being higher for male dominated industrial sectors. These are engineering and technology oriented sectors, compared to administration, services, and education which are the female dominated sectors. The depreciation does not appear to vary across occupational groupings - male dominated and female dominated occupation groupings have similar depreciation rates. These findings need to be treated as preliminary findings as they are only point estimates of depreciation, evaluated at mean values. 

\begin{table}[H]
	\centering 
	\caption{Average Human Capital Depreciation Rates (DR) by Female- and Male-dominated Industries and Occupations, RLMS 2018} 
	\label{tab:3.3} 
	\begin{tabular}{clcccc}
		\hline
		& \textbf{Statistic} &\textbf{Ind\_F}& \textbf{Ind\_M} & \textbf{occfemale} & \textbf{occmale} \\ 
		\hline
		1 & Experience, mean  & 23.45 & 22.97 & 21.67 & 23.48 \\ 
		2 & Education, mean & 14.06 & 13.01 & 13.67 & 12.67 \\ 
		\midrule
		3 & DR Experience, \% & 0.89 & 1.82 & 1.55 & 1.40 \\ 
		4 & DR Education, \% & 0.00 & 0.00 & 0.00 & 0.00 \\ 
		5 & DR Human Capital, \% & 0.89 & 1.82 & 1.55 & 1.40 \\ 
		\hline
	\end{tabular}
\end{table} 


\subsection{Depreciation and Occupational Routineness}

In addition to the examination of human capital depreciation rates in gender-dominated industries and occupations, we explore differences in depreciation  between groups generated by using an array of routine and non-routine task content metrics for jobs. This is important in light of discussion about computers and robots taking over routine oriented jobs. In this analysis, we rely on a recent literature of job classification based on task intensity measures  \citet{mihaylov_152._2019}. These measures are based on the textual analysis of description of jobs in the ISCO 08 classification. Each job lists a detailed set of activities or tasks performed as part of the job, and these activities are rated according to whether they are vulnerable to automation in which case they are classified as Routine (R), otherwise they are Non-Routine (NR). Tasks are also classified depending on their Cognitive (C) or Manual (M) requirements; Cognitive taks are further classified as mainly Analytic (A) or Interactive (I). The results is a five-fold classification of tasks, which is subsequently used to develop a set of measures depending on the incidence of these tasks in the job description.

For purpose of this analyis, we use two of these measures. Routine Task Intensity measure (RTI) denotes a score difference between the summed routine task indices and the summed non-routine task indices: $(RC + RM) - (NRA + NRI + NRM)$ - it is a net measure of job routineness or vulnerability to automation. We also use a gross measure that brings together the non-routine task indices: $NRA + NRI + NRM$. Using the k-means clustering technique for the metrics described, we created two respective categorical variables (drti and dnraim) with categories capturing \textit{high, medium,} and \textit{low} manifestations of the features.

Table \ref{tab:3.4} shows the results of comparing depreciation rates between individuals whose jobs invoke routine or non-routine tasks at a high, medium, or low level. The findings suggest that depreciation explained by experience does not differ substantially between people with jobs with varying routine task intensity. The same outcome also applies to workers varying in the degree of non-routine content at their jobs. As with the findings regarding gender, these should be regarded as preliminary findings subject to further analysis. However, it does appear that the automation aspect of technological change may not be affecting the rate of depreciation of skills - both routine and non-routine intensive jobs undergo depreciation, though it is possibly that the underlying causal factors may be different. 

\begin{table}[H]
	\centering
	\caption{Average Human Capital Depreciation Rates (DR) by Routineness Classification, RLMS 2018}
	\label{tab:3.4}
	\begin{tabular}{clccc|cccccc}
		\hline
		& \textbf{Statistic} & \textbf{High} & \textbf{Low} & \textbf{Medium} & \textbf{High} & \textbf{Low} & \textbf{Medium} \\ 
		\hline
& & \multicolumn{3}{c|}{Net Routine Task Intensity} & \multicolumn{3}{c} {Gross Non-Routiness Measure} \\
		\hline
		1 & Measure & drti & drti & drti & dnraim & dnraim & dnraim \\ 
		2 & Experience, mean  & 21.44 & 22.79 & 22.76 & 22.94 & 22.22 & 22.05 \\ 
		3 & Education, mean & 12.86 & 13.67 & 12.8 & 13.66 & 12.76 & 13.02 \\ 
		\hline
		4 & DR Experience, \% & 1.8 & 1.5 & 1.64 & 1.62 & 1.73 & 1.48 \\ 
		5 & DR Education, \% & 0 & 0 & 0 & 0 & 0 & 0 \\ 
		6 & DR Human Capital, \% & 1.8 & 1.5 & 1.64 & 1.62 & 1.73 & 1.48 \\ 
		\hline
	\end{tabular}
\end{table}

\printbibliography

\newpage
\section*{Appendix}
\addcontentsline{toc}{section}{Appendix}%

\setcounter{table}{0}
\renewcommand{\thetable}{A\arabic{table}}


\begin{table}[!htbp] \centering 
\caption{Results of Estimating Human Capital Depreciation for the Female sample, RLMS} 
	\label{tab:A1}
\begin{tabular}{@{\extracolsep{5pt}}lcccccc} 
\\[-1.8ex]\hline 
\hline \\[-1.8ex] 
& \textbf{1994} & \textbf{1998} & \textbf{2003} & \textbf{2006} & \textbf{2012} & \textbf{2018} \\ 
\\[-1.8ex] & (1) & (2) & (3) & (4) & (5) & (6)\\ 
\hline \\[-1.8ex] 
 Constant & 9.725$^{***}$ & 3.786$^{***}$ & 5.464$^{***}$ & 6.946$^{***}$ & 8.133$^{***}$ & 8.767$^{***}$ \\ 
  & (0.381) & (0.322) & (0.301) & (0.247) & (0.186) & (0.242) \\ 
  & & & & & & \\ 
 Educ, years ($S$) & 0.122$^{***}$ & 0.153$^{***}$ & 0.158$^{***}$ & 0.118$^{***}$ & 0.087$^{***}$ & 0.066$^{***}$ \\ 
  & (0.025) & (0.022) & (0.020) & (0.016) & (0.012) & (0.015) \\ 
  & & & & & & \\ 
 Educ X Exper ($TS$) & $-$0.002$^{*}$ & $-$0.002$^{***}$ & $-$0.002$^{**}$ & $-$0.0002 & $-$0.0001 & 0.0004 \\ 
  & (0.001) & (0.001) & (0.001) & (0.001) & (0.0005) & (0.001) \\ 
  & & & & & & \\ 
 Exper ($T$) & 0.074$^{***}$ & 0.080$^{***}$ & 0.055$^{***}$ & 0.013 & 0.020$^{**}$ & 0.020$^{*}$ \\ 
  & (0.019) & (0.016) & (0.015) & (0.013) & (0.010) & (0.011) \\ 
  & & & & & & \\ 
 Exper squared ($T^2$) & $-$0.001$^{***}$ & $-$0.001$^{***}$ & $-$0.001$^{***}$ & $-$0.0003$^{**}$ & $-$0.0005$^{***}$ & $-$0.001$^{***}$ \\ 
  & (0.0002) & (0.0002) & (0.0002) & (0.0001) & (0.0001) & (0.0001) \\ 
  & & & & & & \\ 
\hline \\[-1.8ex] 
Observations & 1,645 & 1,667 & 2,093 & 2,630 & 4,057 & 3,312 \\ 
R$^{2}$ & 0.051 & 0.089 & 0.110 & 0.139 & 0.104 & 0.092 \\ 
Adjusted R$^{2}$ & 0.049 & 0.087 & 0.108 & 0.138 & 0.103 & 0.091 \\ 
Residual Std. Error & 0.853 & 0.728 & 0.731 & 0.664 & 0.641 & 0.597 \\ 
F Statistic & 22.179$^{***}$ & 40.520$^{***}$ & 64.342$^{***}$ & 106.385$^{***}$ & 117.366$^{***}$ & 83.993$^{***}$ \\ 
\hline 
\hline \\[-1.8ex] 
\textit{Note:}  & \multicolumn{6}{r}{$^{*}$p$<$0.1; $^{**}$p$<$0.05; $^{***}$p$<$0.01} \\ 
\end{tabular} 
\end{table} 


\begin{table}[!htbp] \centering 
\caption{Results of Estimating Human Capital Depreciation for the Male sample, RLMS} 
	\label{tab:A2}
\begin{tabular}{@{\extracolsep{5pt}}lcccccc} 
\\[-1.8ex]\hline 
\hline \\[-1.8ex] 
& \textbf{1994} & \textbf{1998} & \textbf{2003} & \textbf{2006} & \textbf{2012} & \textbf{2018} \\ 
\\[-1.8ex] & (1) & (2) & (3) & (4) & (5) & (6)\\ 
\hline \\[-1.8ex] 
 Constant & 10.357$^{***}$ & 5.029$^{***}$ & 7.334$^{***}$ & 8.067$^{***}$ & 8.771$^{***}$ & 9.094$^{***}$ \\ 
  & (0.433) & (0.360) & (0.282) & (0.243) & (0.157) & (0.185) \\ 
  & & & & & & \\ 
 Educ, years ($S$) & 0.136$^{***}$ & 0.123$^{***}$ & 0.080$^{***}$ & 0.077$^{***}$ & 0.077$^{***}$ & 0.077$^{***}$ \\ 
  & (0.028) & (0.024) & (0.019) & (0.016) & (0.010) & (0.012) \\ 
  & & & & & & \\ 
 Educ X Exper ($TS$) & $-$0.002$^{*}$ & $-$0.001 & 0.0004 & $-$0.0003 & $-$0.0004 & $-$0.001 \\ 
  & (0.001) & (0.001) & (0.001) & (0.001) & (0.0005) & (0.001) \\ 
  & & & & & & \\ 
 Exper ($T$) & 0.054$^{**}$ & 0.032$^{*}$ & 0.002 & 0.007 & 0.035$^{***}$ & 0.037$^{***}$ \\ 
  & (0.023) & (0.017) & (0.014) & (0.013) & (0.009) & (0.010) \\ 
  & & & & & & \\ 
 Exper squared ($T^2$) & $-$0.001$^{***}$ & $-$0.0004$^{**}$ & $-$0.0003$^{*}$ & $-$0.0003$^{*}$ & $-$0.001$^{***}$ & $-$0.001$^{***}$ \\ 
  & (0.0003) & (0.0002) & (0.0002) & (0.0001) & (0.0001) & (0.0001) \\ 
  & & & & & & \\ 
\hline \\[-1.8ex] 
Observations & 1,392 & 1,433 & 1,763 & 2,170 & 3,360 & 2,800 \\ 
R$^{2}$ & 0.057 & 0.070 & 0.078 & 0.074 & 0.153 & 0.110 \\ 
Adjusted R$^{2}$ & 0.054 & 0.067 & 0.076 & 0.072 & 0.152 & 0.108 \\ 
Residual Std. Error & 0.951 & 0.803 & 0.754 & 0.688 & 0.598 & 0.570 \\ 
F Statistic & 20.989$^{***}$ & 26.879$^{***}$ & 37.362$^{***}$ & 43.281$^{***}$ & 151.868$^{***}$ & 86.125$^{***}$ \\ 
\hline 
\hline \\[-1.8ex] 
\textit{Note:}  & \multicolumn{6}{r}{$^{*}$p$<$0.1; $^{**}$p$<$0.05; $^{***}$p$<$0.01} \\ 
\end{tabular} 
\end{table} 

\newpage

\end{document}
