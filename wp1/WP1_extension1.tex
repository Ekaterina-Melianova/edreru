\documentclass[12pt,a4paper]{article}

\usepackage[labelfont=bf]{caption}
\usepackage[font=small,labelfont=bf]{subcaption}
\usepackage{tikz}

\usepackage[backend=biber, sorting=none,  style=authoryear]{biblatex}
\addbibresource{Returns.bib}

\usepackage{float}

\usepackage[T1]{fontenc}

\usepackage[sc]{mathpazo}
\usepackage{geometry}
\geometry{verbose,tmargin=2.5cm,bmargin=2.5cm,lmargin=2.5cm,rmargin=2.5cm}
\setcounter{secnumdepth}{2}
\setcounter{tocdepth}{2}
\usepackage{url}
\usepackage[unicode=true,pdfusetitle,
 bookmarks=true,bookmarksnumbered=true,bookmarksopen=true,bookmarksopenlevel=2,
 breaklinks=false,pdfborder={0 0 1},backref=false,colorlinks=false]
 {hyperref}
\hypersetup{
 pdfstartview={XYZ null null 1}}
\usepackage{breakurl}

\usepackage{Sweave}
\begin{document}
\Sconcordance{concordance:WP1_extension1.tex:WP1_extension1.Rnw:%
1 27 1 1 0 6 1 1 2 1 0 14 1 3 0 1 2 1 3 2 0 1 1 1 3 1 0 1 5 3 0 1 6 4 0 %
1 3 1 0 1 5 3 0 1 3 1 0 5 1 1 2 1 0 1 1 1 3 2 0 1 1 1 2 1 3 2 0 1 3 1 0 %
1 2 1 0 1 3 1 0 2 1 1 5 3 0 1 3 2 0 1 1 1 11 9 0 1 3 1 0 1 3 1 0 1 3 1 %
0 1 8 7 0 1 2 2 1 1 5 3 0 1 5 3 0 1 3 1 0 1 4 2 0 1 9 8 0 1 4 3 0 1 3 1 %
0 1 1 1 4 2 0 1 6 4 0 1 1 4 0 1 3 1 4 2 0 1 1 1 4 3 0 1 17 14 0 2 1 3 0 %
1 2 1 34 35 0 1 2 17 1}


\title{WP1 extension along lines of Neuman-Weiss 1995}
\author{Suhas D. Parandekar, Ekaterina Melianova, Art\"{e}m Volgin}
\maketitle

\begin{Schunk}
\begin{Sinput}
> library(dplyr)
> library(sqldf)
> library(XLConnectJars)
> library(questionr)
> library(labelled)
> library(tidyr)
> library(magrittr)
> library(ggplot2)
> library(data.table)
> library(pbapply)
> library(gridExtra)
> library(psych)
> library(stringi)
> library(sjPlot)
> library(sjmisc)
\end{Sinput}
\end{Schunk}

\begin{Schunk}
\begin{Sinput}
> # wd
> wd <- "C:/Country/Russia/Data/SEASHELL/SEABYTE/edreru/wp1"
> setwd(wd)
> # Data
> df_mincer <- readRDS("df_mincer.rds")
> ########## Aggregating occupations by 2 digits
> 
> # First, aggregating military men
> df_mincer$occup <- ifelse(df_mincer$occup == 0, 110, df_mincer$occup) 
> # For simplicity those with one digit go to the first category with 2
> # digits (e.g., 1 go to 11)
> df_mincer$occup <- as.numeric(ifelse(df_mincer$occup<10,
+                                      paste0(df_mincer$occup, "1"),
+                                      df_mincer$occup))
> # Leaving only 2 digits
> df_mincer$occup2d <- as.numeric(substr(df_mincer$occup, 1, 2))
> # If N obs is < 30 we technically cannot run a regression
> # Let's aggregate such categories with the respective closest category
> # I detected them manually for the easiness of computations based on this table:
> table(df_mincer$occup2d, df_mincer$YEAR)
> # Aggregating
> df_mincer$occup2d[df_mincer$occup2d == 11|df_mincer$occup2d == 12] <- 112
> df_mincer$occup2d[df_mincer$occup2d == 24|df_mincer$occup2d == 25] <- 245
> df_mincer$occup2d[df_mincer$occup2d == 34|df_mincer$occup2d == 35] <- 345
> df_mincer$occup2d[df_mincer$occup2d == 41|df_mincer$occup2d == 42] <- 412
> df_mincer$occup2d[df_mincer$occup2d == 43|df_mincer$occup2d == 44] <- 434
> df_mincer$occup2d[df_mincer$occup2d == 53|df_mincer$occup2d == 54] <- 534
> df_mincer$occup2d[df_mincer$occup2d == 73|df_mincer$occup2d == 74|
+                   df_mincer$occup2d == 75] <- 7345
> df_mincer$occup2d[df_mincer$occup2d == 81|df_mincer$occup2d == 82] <- 812
> df_mincer$occup2d[df_mincer$occup2d == 92|df_mincer$occup2d == 93|
+                   df_mincer$occup2d == 94|df_mincer$occup2d == 95|
+                   df_mincer$occup2d == 96] <- 923456
> df_mincer$occup2d[df_mincer$occup2d == 61|df_mincer$occup2d == 62] <- 71
> table(df_mincer$occup2d, df_mincer$YEAR)
> # 345 category is too small even within its digit so 
> # we need to merge it with another digit
> df_mincer$occup2d[df_mincer$occup2d == 345] <- 412
> # Checking
> tbl <- as.data.frame(table(df_mincer$occup2d, df_mincer$YEAR))
> tiny <- as.numeric(as.character(
+   unique(tbl$Var1[tbl$Freq<30]))); tiny # no categaries with < 30 obs
> # Creating a dummy set for occupations
> dummy_set <- dummy.code(df_mincer$occup2d)
> colnames(dummy_set) <- paste0("occup", colnames(dummy_set), sep = "")
> df <- cbind(df_mincer, dummy_set)
> # Probit regression: developing a female - non-female typology of occupations
> 
> # Empty list where the regression output will be written
> probit <- vector("list", length(unique(df$YEAR)))
> for (i in seq(length(probit))){
+   probit[[i]] <- vector("list", length(unique(colnames(dummy_set))))
+ }
> seq_year <- unique(df$YEAR)
> # Looping over each year and occupation
> # takes ~15 sec
> for(i in seq(length(seq_year))){
+   for(j in seq(length(colnames(dummy_set)))){
+     probit[[i]][[j]] <- glm(as.formula(paste0(colnames(dummy_set)[j], "~",
+                                               "female")),
+                           family = binomial(link = "probit"),
+                           data = df[df$YEAR == seq_year[i],])
+   }
+ }
> # Naming
> names(probit) <- seq_year
> # Computng summary
> smry <-  lapply(probit, function(x) {lapply(x, summary)})
> # A table with coefficients for the female variable
> tbl_fem_ <- c()
> for (y in seq_year){
+   for (n in seq(length(colnames(dummy_set)))){
+    tbl_fem_ <- rbind.data.frame(tbl_fem_, cbind.data.frame(
+     "female" = round(smry[[paste0(y)]][[n]]$coefficients[2,1], 2),
+     "p-value" = round(smry[[paste0(y)]][[n]]$coefficients[2,4], 3))
+     )   
+   }
+ }
> YEAR <- rep(seq_year, each = ncol(dummy_set))
> occup <- rep(colnames(dummy_set), length(seq_year))
> tbl_fem <- cbind.data.frame(YEAR, occup, tbl_fem_) # final table
> # Female-dominated occupations
> fem_occup_vec <- as.character(unique(
+   tbl_fem[tbl_fem$female > 0 &
+             tbl_fem$'p-value' < 0.05, "occup"]))
> # Non-female occupations
> nonfem_occup_vec <- as.character(unique(
+   tbl_fem[!(tbl_fem$female > 0 &
+             tbl_fem$'p-value' < 0.05), "occup"]))
> # Occupations which are in both categories depending on a wave
> fem_occup_vec[fem_occup_vec %in% nonfem_occup_vec]
> # Let us examine those cases
> both <- tbl_fem[tbl_fem$occup %in% 
+                   fem_occup_vec[fem_occup_vec %in% nonfem_occup_vec],]
> # occup14 is insignificant in the majoriy of waves -> let's put it in
> # nonfem_occup
> # occup245 is significant and positive almost each time -> let's put it in
> # fem_occup
> # occup51 is mostly significant and positive -> let's put it in fem_occup
> # occup91 is mostly significant and positive -> let's put it in fem_occup
> 
> # Defining a variable with nonfem_occup
> df$occup2d <- as.character(df$occup2d)
> df$fem_occup <- ifelse(df$occup2d %in% 
+                             substr(fem_occup_vec[!fem_occup_vec=="occup14"],
+                                    6, nchar(fem_occup_vec[
+                                      !fem_occup_vec=="occup14"])), 1, 0)
> # Looking at the distribution (looks logical)
> table(df$fem_occup)
> table(df$fem_occup, df$female)
> # Filtering the missings left
> df <- df %>%
+   filter(!is.na(wage) & !is.na(edu_4) & wage > 0)
> # Norming wages (to allow comparison)
> # wages_prop <- rio::import("wages_prop.xlsx") # Rosstat statistics
> df <- df %>%
+   group_by(YEAR) %>%
+   mutate(wageBYmed = wage/median(wage))
> aggregate(wageBYmed ~ YEAR, df, mean)
> 
\end{Sinput}
\end{Schunk}

\begin{Schunk}
\begin{Sinput}
> # Empty list where the regression output will be written
> lm_dep <- vector("list", length(unique(df$YEAR)))
> seq_year <- unique(df$YEAR)
> df$fem_occup <- factor(df$fem_occup, 
+                           levels = c(1,0),
+                           labels = c("Female Occupations",
+                                      "Non-female Occupations"))
> # Looping over each year
> for(i in seq(length(seq_year))){
+   lm_dep[[i]] <- lm(log(wageBYmed) ~ edu_4 +
+                                 exper + 
+                                 I(exper^2) + 
+                                 fem_occup +
+                                 fem_occup*exper +
+                                 fem_occup*I(exper^2) +
+                                 fem_occup*edu_4 +
+                                 exper*edu_4 +
+                                 I(exper^2)*edu_4 +
+                                 fem_occup*edu_4*exper +
+                                 fem_occup*edu_4*I(exper^2),
+                            data = df[df$YEAR == seq_year[i],])
+ }
> names(lm_dep) <- seq_year
> smry_lm_dep <- lapply(lm_dep, summary)
\end{Sinput}
\end{Schunk}

\begin{Schunk}
\begin{Sinput}
> ################################# Model prediction
> for (i in 1:length(seq_year)){
+   df_year <- as.data.frame(df[df$YEAR == seq_year[i],])
+   pred_y <- exp(predict(lm_dep[[i]], df_year, interval="conf"))
+   df_year <- cbind(df_year, pred_y)
+   
+ # Plot
+  p_int <- ggplot(df_year, aes(x = exper, y = fit, group = edu_4, color = edu_4,
+                   linetype = edu_4)) +
+     geom_line(aes(y = fit), size = 1.2) +
+     geom_ribbon(aes(ymin=lwr, ymax=upr, fill = edu_4), alpha = 0.1, 
+                 colour = NA) +
+     facet_grid(~ as.factor(fem_occup)) +
+     theme(legend.title = element_blank(),
+          legend.position = "bottom",
+          panel.grid.minor = element_blank(),
+          axis.text.x = element_text(size = 12),
+          axis.text.y = element_text(size = 12),
+          axis.title = element_text(size = 12),
+          legend.text = element_text(size = 12),
+          legend.key = element_rect(size = 12))  +
+     scale_color_manual(values = c("blue", "red", "darkgreen")) + 
+     scale_fill_manual(values=c("blue", "red", "darkgreen")) +
+     scale_linetype_manual(values = c("solid", "longdash", "dotted")) +
+     scale_y_continuous(limits = c(0, 3)) +
+     ylab("Monthly wage normed by median") +
+     xlab("Experience") 
+   
+  ggsave(paste0("p_", seq_year[i], "_int.png"), height = 4, width = 7.5,
+         units = "in")
+  print(i)
+ }
\end{Sinput}
\end{Schunk}


\foreach \year in {1994, 1995, 1996, 1998, 2000, 2001,
                   2002, 2003, 2004, 2005, 2006, 2007,
                   2008, 2009, 2010, 2011, 2012, 2013,
                   2014, 2015, 2016, 2017, 2018} {
    \begin{figure}[ht]
    \begin{center}
      \centering
      %\hspace*{-0.8in}
      \includegraphics[width=17cm,height=8cm]{p_\year_int.png}
      \caption{Experience Profiles, RLMS \year}
      %\label{fig:1}
    \end{center}
    \end{figure}
 }

\end{document}
